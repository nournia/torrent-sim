\documentclass{article}
\usepackage{graphicx}
\usepackage{amsmath}
\usepackage{xepersian}
\settextfont[Scale=1.2]{XB Zar}

% section numbering
\setcounter{secnumdepth}{3}
\renewcommand{\thesection}{\arabic{section}}
\renewcommand{\thesubsection}{\thesection.\arabic{subsection}}

\title{
\begin{normalsize} به نام خدا \end{normalsize}
\\[2cm]
 مدل‌سازی پروتکل بیت‌تورنت با استفاده از سیستم چند عامله
}
\author{علیرضا نوریان
\\
\\ \small دانشگاه علم و صنعت ایران
\\ \small nourian@comp.iust.ac.ir
}

\begin{document}
\maketitle

\section{مقدمه}
شبکه‌های نظیر به نظیر\LTRfootnote{Peer to Peer} در سالهای اخیر بسیار مشهور شدند. استفاده‌ی عمده از این شبکه‌ها در به اشتراک گذاری فایلها و برقراری تماس میان کاربران است. در چنین شبکه‌ای به جای یک سرور و چند کلاینت، همه‌ی اعضا هم سرور هستند و هم کلاینت که نتیجه، برداشتن بار سنگین خدمات از دوش سرور و بالا رفتن سرعت دانلود برای همه کلاینتها خواهد بود. با نگاه عامل به کلاینت‌ها می‌توان چنین سیستمی را به خوبی تحلیل کرد چراکه همکاری \LTRfootnote{cooperate} همه‌ی عاملها موجب بهره‌ی فردی برای هر کدام می‌شود. از بزرگترین چالشهای پیش‌رو برای پروتکلهای اشتراک فایل نظیر به نظیر، تشویق عاملها به همکاری (آپلود کردن آنچه از دیگران دانلود کرده‌اند) است. پروتکل‌های زیادی برای چنین سیستمی پیشنهاد شده‌اند و بیت‌تورنت\LTRfootnote{BitTorrent} هم یکی از آنهاست که می‌توان گفت مشکل همکاری میان عاملها را حل کرده است.

برای مدل‌سازی شبکه‌های نظیر به نظیر رهیافتهای گوناگونی مورد استفاده قرار گرفته است. گزارش پیش‌رو حاصل پیاده‌سازی یکی از این تجربه‌ها\cite{mas-torrent} است که به هر کلاینت به عنوان یک عامل نگاه می‌کند و او را خودمختار\LTRfootnote{Autonomous}، اجتماعی\LTRfootnote{Sociable}، واکنشی\LTRfootnote{Reactive} و فعال\LTRfootnote{Pro-Active} می‌داند. استفاده از مدل چند عامله امکان شبیه‌سازی رفتارهای زیر را در مدل فراهم می‌کند:
\begin{itemize}
\item
اختصاص دادن رفتارهای گوناگون به کاربران
\item
اختصاص پهنای باند آپلود و دانلود متفاوت به هر کاربر
\item
قرار دادن افرادی که تنها به نفع خود فکر می‌کنند و به دیگران کمک نمی‌کنند
\item
تعیین نحوه ورود و خروج کاربران به شبکه
\end{itemize}

در ادامه پس از نگاهی اجمالی به سیستم‌های نظیر به نظیر و سیستم‌های چند عامله، مدل پروتکل بیت‌تورنت و نحوه‌ی پیاده‌سازی آن را شرح داده و در نهایت نتایج تجربی را مشاهده می‌کنیم. 

\subsection{سیستم نظیر به نظیر}
سیستم‌های نظیر به نظیر مزایا و البته چالشهای فراوانی دارند. معمولا این سیستم‌ها در مقابل بزرگ شدن ابعاد کاملا انعطاف‌پذیرند و خطاهای مربوط به خرابی در بخشهای مختلف شبکه به مقاومت آنها ضربه‌ای نمی‌زند. در چنین سیستمی، مدیریت، نگهداری و حتی مالکیت کل سیستم میان کاربران توزیع شده که منجر به تسهیل دسترسی به منابع برای همه می‌شود.

ویژگی به اشتراک گذاری بهینه‌ی منابع در این سیستم‌ها موجب استفاده از آنها در سیستم‌های توزیع پردازش، پایگاه‌داده‌های توزیع‌شده، سیستم‌های توزیع محتوی و ... شده است. در یک سیستم نظیر به نظیر، هر کاربر ممکن است با قرار ندادن منابع خود در اختیار دیگران، برای حفظ منافع شخصی خود، از همکاری سرباز زند و فقط سعی کند از دیگران خدمات بگیرد و این دقیقا همان مشکلی است که در بسیاری از سیستم‌های چند عامله با آن دست به گریبان هستیم.  

\subsection{سیستم چند عامله}
اگرچه اتفاق نظری سراسری در مورد مفهوم عامل\LTRfootnote{Agent} وجود ندارد، اما «سخت‌افزار یا نرم‌افزاری که خودمختار، اجتماعی، واکنشی و فعال باشد»\cite{wooldridge} تا حدی می‌تواند منظور ما را از عامل روشن کند. اما هنگامی که از یک سیستم چند عامله صحبت می‌کنیم، گروهی از عامل‌ها را مد نظر داریم که با هم تعامل می‌کنند و حیات جمعی آنها با این تعامل معنا می‌پذیرد.

سیستم‌های چند عامله افق‌های تازه‌ای را پیش‌روی طراحان سیستم‌ها قرار داده‌اند. در این نگاه یک سیستم پیچیده را می‌توان به صورت اجزای ساده‌ای دید که هر کدام به دنبال دستیابی به اهداف خود هستند و مجموعه‌ی آنها اهداف طراح کل سیستم را پی می‌گیرد. چنین سیستمی می‌تواند به صورت بسیار وسیع، مقاوم، اتکاپذیر و البته منعطف و به صرفه طراحی شود. این ویژگی‌ها منجر به استفاده از تکنولوژی سیستم‌های چند عامله در داده‌کاوی و بازیابی اطلاعات توزیع‌شده، تجارت الکترونیک، ارتباطات و ... شده است.

چارچوب JADE برای پیاده‌سازی یک سیستم چندعامله طراحی شده است و نویسندگان اصلی مقاله از آن برای مدل‌سازی استفاده کرده‌اند. اما در این گزارش که تلاشی برای پیاده‌سازی بیت‌تورنت به صورت بسیار ساده است، از چاچوب SPADE\cite{spade} برای این منظور استفاده شد. دلیل اصلی این انتخاب نیز مبتنی بودن این چارچوب بر پایتون است که بی‌تردید انتخاب بسیار بهتری از جاواست. این چارچوب امکان ارتباط آسان میان عامل‌ها را فراهم کرده و اجرای آنها را در یک شبکه ممکن می‌سازد. همچنین در طراحی این چارچوب امکان استفاده از منطق مرتبه اول نیز در نظر گرفته شده است. 

\section{مدل پروتکل بیت‌تورنت}

\subsection{مقدمه}
پروتکل بیت‌تورنت به‌گونه‌ای تنظیم شده است که انتشار فایل‌ها از طریق معامله‌ی قطعات کوچک فایل میان نظیرها صورت گیرد. همانطور که اشاره شد، در این پروتکل یک واحد مرکزی که شاهد وقوع همه‌ی اتفاقات است، وجود ندارد. برای تحقق اشتراک فایل بدون واحد مرکزی سه عنصر زیر تعریف شده‌اند:

\subsubsection{فایل تورنت}
برای اشتراک هر فایلی با این پروتکل باید یک فایل تورنت\LTRfootnote{Torrent File} برای توصیف آن ایجاد شود. در این فایل فهرستی از قطعات فایل ذکر شده و هر قطعه در آن با یک کد مشخص می‌شود. این کد از اجرای الگوریتم SHA1 روی محتوی قطعه به دست آمده و از آن علاوه بر تمایز قطعات، برای تایید صحت محتوی قطعه دریافت‌شده نیز استفاده می‌شود.

\subsubsection{\LTRfootnote{Tracker}واسط}
در فایل تورنت علاوه بر قطعات فایل اصلی، آدرس دستیابی به واسط دریافت (معمولا یک سرور وب) نیز ذکر شده است. ورود به شبکه‌ی اشتراک یک فایل معمولا با پرسش دارندگان قطعات فایل از این واسط آغاز می‌شود. همچنین کلاینت‌ها به صورت متناوب از این واسط، فهرست آخرین افراد شبکه را می‌پرسند و به آن وضعیت خود را اطلاع می‌دهند. سیاست اصلی آن است که ورود به شبکه با برقراری ارتباط با این واسط آغاز می‌شود ولی بعد از ورود به شبکه نیاز حیاتی به اتصال با واسط وجود ندارد.

\subsubsection{کلاینت}
فایل اصلی برای توزیع باید در اختیار چند کلاینت قرار بگیرد، تا آنها بتوانند نقش توزیع‌کننده‌های اولیه را ایفا کنند. با تشکیل شبکه، کم کم بقیه کلاینت‌ها نیز مالک قطعات فایل شده و در نهایت به مالکان کل فایل تبدیل می‌شوند.

به طور خلاصه می‌توان گفت کلاینت برای دانلود فایل ابتدا باید فایل تورنت مرتبط با آن را دریافت کرده و با سرور واسط ارتباط برقرار کند. سپس با دریافت قطعات اولیه شروع به معامله قطعات کرده و در این هنگام می‌تواند سرعت بالای دریافت فایل را تجربه کند. 

\subsection{فرایندها}
برای داشتن یک شبکه‌ی کارا، ترتیب تقاضای قطعات فایل توسط کلاینت‌ها باید مورد توجه قرار گیرد. تلاش بر این است هر کلاینت قطعاتی را دریافت کند کمتر در دسترس هستند، تا کلاینت‌ها به قطعات همدیگر علاقمند باشند. در غیر این صورت ممکن است در پایان فرایند همه‌ی کلاینت‌ها قطعات خاصی را داشته باشند ولی بعضی از قطعات در اختیار هیچ کس نباشند. 

وقتی کلاینتی وارد شبکه می‌شود، چیزی برای معامله ندارد به همین دلیل هر قطعه‌ای را که بتواند، دریافت می‌کند. اصطلاحا کلاینت در وضعیت «دریافت اتفاقی» قرار دارد. این وضع تا وقتی ادامه پیدا می‌کند که ۴ قطعه دریافت کند. پس از این مرحله کلاینت به وضعیت «دریافت قطعات نادر» می‌رود که در آن تلاش می‌کند قطعاتی که افراد کمتری مالک آن هستند را دریافت کند. وقتی که تنها چند قطعه برای دریافت باقیمانده، کلاینت در وضعیت «پایان دریافت» قرار می‌گیرد و در این حال هر آنچه را که نیاز دارد از چندین مالک دریافت می‌کند. البته به محض دریافت قطعه از یک طریق، تقاضاهای دیگر را لغو می‌کند.

فرایند معامله در این پروتکل از مهمترین ویژگی‌های آن است. در این فرایند با اقتباس از مفهوم تلافی در نظریه‌ی بازی‌ها، کلاینت‌ها سعی می‌کنند اجازه‌ی دانلود را بیشتر به افرادی بدهند که آنها نیز این امکان را به طور متقابل فراهم کرده‌اند. البته این تفکر فقط برای کلاینت‌هایی است که نیاز به دریافت فایل دارند و آنها که فایل را دانلود کرده‌اند، طور دیگری تصمیم می‌گیرند.

\section{نتایج}
آنچه که واقعا پیاده‌سازی شد


\renewcommand*{\refname}{\section{منابع}}
\begin{thebibliography}{9}
\begin{latin}

\bibitem{mas-torrent}
E. Costa-Montenegro, J. Burguillo-Rial, F. Gil-Castiñeira, and F. González-Castaño, “Implementation and analysis of the BitTorrent protocol with a multi-agent model,” Journal of Network and Computer Applications, vol. 34, no. 1, pp. 368–383, 2011.

\bibitem{wooldridge}
M. Wooldridge and N. R. Jennings, “Intelligent agents: Theory and practice,” Knowledge engineering review, vol. 10, no. 2, pp. 115–152, 1995.

\bibitem{spade}
M. E. Gregori, J. P. Cámara, and G. A. Bada, “A jabber-based multi-agent system platform,” in Proceedings of the fifth international joint conference on Autonomous agents and multiagent systems, 2006, pp. 1282–1284.

\end{latin}
\end{thebibliography}

\end{document}
