\documentclass{article}
\usepackage{graphicx}
\usepackage{amsmath}
\usepackage{xepersian}
\settextfont[Scale=1.2]{XB Zar}

% section numbering
\setcounter{secnumdepth}{3}
\renewcommand{\thesection}{\arabic{section}}
\renewcommand{\thesubsection}{\thesection.\arabic{subsection}}

\title{ 
\begin{normalsize} به نام خدا \end{normalsize}
\\[2cm]
 مدل‌سازی پروتکل بیت‌تورنت با استفاده از سیستم چند عامله
}
\author{علیرضا نوریان
\\
\\ \small دانشگاه علم و صنعت ایران
\\ \small nourian@comp.iust.ac.ir
}

\begin{document}
\maketitle

\section{مقدمه}

\subsection{سیستمهای نظیر به نظیر}

\subsection{سیستم چند عامله}

\subsection{SPADE}

\section{مدل پروتکل بیت‌تورنت}

\section{نتایج}


\renewcommand*{\refname}{\section{منابع}}
\begin{thebibliography}{9}
\begin{latin}

\bibitem{mas-torrent}
E. Costa-Montenegro, J. Burguillo-Rial, F. Gil-Castiñeira, and F. González-Castaño, “Implementation and analysis of the BitTorrent protocol with a multi-agent model,” Journal of Network and Computer Applications, vol. 34, no. 1, pp. 368–383, 2011.

\bibitem{spade}

\end{latin}
\end{thebibliography}

\end{document}
